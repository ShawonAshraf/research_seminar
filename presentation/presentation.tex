%%%%%%%%%%%%%%%%%%%%%%%%%%%%%%%%%%%%%%%%%
% Beamer Presentation
% LaTeX Template
% Version 1.0 (10/11/12)
%
% This template has been downloaded from:
% http://www.LaTeXTemplates.com
%
% License:
% CC BY-NC-SA 3.0 (http://creativecommons.org/licenses/by-nc-sa/3.0/)
%
%%%%%%%%%%%%%%%%%%%%%%%%%%%%%%%%%%%%%%%%%

%----------------------------------------------------------------------------------------
%	PACKAGES AND THEMES
%----------------------------------------------------------------------------------------

\documentclass{beamer}

\mode<presentation> {

% The Beamer class comes with a number of default slide themes
% which change the colors and layouts of slides. Below this is a list
% of all the themes, uncomment each in turn to see what they look like.

%\usetheme{default}
%\usetheme{AnnArbor}
%\usetheme{Antibes}
%\usetheme{Bergen}
%\usetheme{Berkeley}
\usetheme{Berlin}
%\usetheme{Boadilla}
%\usetheme{CambridgeUS}
%\usetheme{Copenhagen}
%\usetheme{Darmstadt}
%\usetheme{Dresden}
%\usetheme{Frankfurt}
%\usetheme{Goettingen}
%\usetheme{Hannover}
%\usetheme{Ilmenau}
%\usetheme{JuanLesPins}
%\usetheme{Luebeck}
%\usetheme{Madrid}
%\usetheme{Malmoe}
%\usetheme{Marburg}
%\usetheme{Montpellier}
%\usetheme{PaloAlto}
%\usetheme{Pittsburgh}
%\usetheme{Rochester}
%\usetheme{Singapore}
%\usetheme{Szeged}
%\usetheme{Warsaw}

% As well as themes, the Beamer class has a number of color themes
% for any slide theme. Uncomment each of these in turn to see how it
% changes the colors of your current slide theme.

%\usecolortheme{albatross}
%\usecolortheme{beaver}
%\usecolortheme{beetle}
%\usecolortheme{crane}
%\usecolortheme{dolphin}
%\usecolortheme{dove}
%\usecolortheme{fly}
%\usecolortheme{lily}
%\usecolortheme{orchid}
%\usecolortheme{rose}
%\usecolortheme{seagull}
%\usecolortheme{seahorse}
%\usecolortheme{whale}
%\usecolortheme{wolverine}

%\setbeamertemplate{footline} % To remove the footer line in all slides uncomment this line
%\setbeamertemplate{footline}[page number] % To replace the footer line in all slides with a simple slide count uncomment this line

%\setbeamertemplate{navigation symbols}{} % To remove the navigation symbols from the bottom of all slides uncomment this line
}

\usepackage{graphicx} % Allows including images
\usepackage{booktabs} % Allows the use of \toprule, \midrule and \bottomrule in tables
\usepackage{hyperref}
\usepackage{xcolor}
\usepackage{amsmath}


%----------------------------------------------------------------------------------------
%	TITLE PAGE
%----------------------------------------------------------------------------------------

\title[Research Seminar, Winter 2021-2022]{GQA: A New Dataset for Real-World Visual Reasoning and Compositional Question Answering. Hudson et al. (2019)} % The short title appears at the bottom of every slide, the full title is only on the title page

\author{Shawon Ashraf} % Your name
\institute[IMS, Universität Stuttgart] % Your institution as it will appear on the bottom of every slide, may be shorthand to save space
{
Research Seminar, Computational Linguistics \\
Winter Semester 2021-2022 \\ % Your institution for the title page
Universität Stuttgart \\
}
\date{16 December 2021} % Date, can be changed to a custom date

\begin{document}

\begin{frame}
\titlepage % Print the title page as the first slide
\end{frame}

\begin{frame}
\frametitle{Overview} % Table of contents slide, comment this block out to remove it
\tableofcontents % Throughout your presentation, if you choose to use \section{} and \subsection{} commands, these will automatically be printed on this slide as an overview of your presentation
\end{frame}

%----------------------------------------------------------------------------------------
%	PRESENTATION SLIDES
%----------------------------------------------------------------------------------------

%------------------------------------------------
\section{Introduction and Background} % Sections can be created in order to organize your presentation into discrete blocks, all sections and subsections are automatically printed in the table of contents as an overview of the talk
%------------------------------------------------

\begin{frame}
\frametitle{What is GQA?}

\begin{itemize}
    \item Visual Question Answering
        \begin{itemize}
            \item \textbf{Understanding} a Scene (Image) and \textbf{answering questions} on it
            \item Semantic Reasoning
        \end{itemize}
    \item Combination of Structural and Semantic Information
        \begin{itemize}
            \item \textbf{Objects} in the image and how they are \textbf{related}
            \item How the objects and relations connect to question text
        \end{itemize}
    \item Scene Graphs
        \begin{itemize}
            \item Data Structure for the required \textbf{structure}
            \item Source: Visual Genome \cite{vgenome}
        \end{itemize}
\end{itemize}

\end{frame}

\begin{frame}
\frametitle{Consider this image (Source: GQA)}
\begin{figure}
\includegraphics[width=0.8\linewidth]{gqa_1.png}
\end{figure}
\end{frame}


\begin{frame}
\frametitle{A scene graph of the previous image (Source: GQA)}
\begin{figure}
\includegraphics[width=0.8\linewidth]{scene_graph.png}
\end{figure}
\end{frame}


\begin{frame}{Why do we need it?}
    \begin{itemize}
        \item \textbf{Lack of diveristy in the labels} in existing datasets (CLEVR \cite{clevr}, VQA \cite{vqa} etc.)
        \item Susceptible to \textbf{lucky guesses}
        \item Existing approaches (such as CNN, LSTM etc.) \textbf{can easily game} these datasets
        \item GQA solves this problem 
    \end{itemize}
\end{frame}
%------------------------------------------------

\section{Dataset}

\begin{frame}
\frametitle{Example Scene Graph Stucture in GQA}
\begin{figure}
\includegraphics[width=0.7\linewidth]{sg_json.png}
\end{figure}
\end{frame}

\begin{frame}{Statistics}
   \begin{table}[H]
        \begin{tabular}{@{}lll@{}}
            \toprule
            Questions  & Images  & Labels \\ \midrule
            22,669,678 & 113,018 & 1878 \\  \bottomrule
        \end{tabular}
        \caption{Questions, Images and Labels in GQA}
    \end{table}
    
    \begin{block}{Evaluation settings}
        \begin{itemize}
            \item Test Size: 5000; \textbf{Crowd Sourced}
            \item Train:Validation $\rightarrow$ $88:12$
        \end{itemize}
    \end{block}
    
\end{frame}


%------------------------------------------------

\section{Results}
\begin{frame}{Accuracy}
\begin{figure}
\includegraphics[width=0.7\linewidth]{presentation/acc_plot.png}
\end{figure}
\end{frame}
%------------------------------------------------

\section{Conclusion}
\begin{frame}{Concluding GQA}
    \begin{itemize}
        \item Benchmark for testing semantic reasoning in models
        \item Posted as a challenge task: \hyperlink{https://cs.stanford.edu/people/dorarad/gqa/challenge.html}{https://cs.stanford.edu/people/dorarad/gqa/challenge.html}
    \end{itemize}
\end{frame}
%------------------------------------------------

%------------------------------------------------
% ----------------------------- Ref Section -----------------------------------------------------------------
\section{References}

\begin{frame}
%\frametitle{References}
\footnotesize{

\begin{thebibliography}{99} % Beamer does not support BibTeX so references must be inserted manually as below
\bibitem[Johnson et al, 2016]{clevr} Justin Johnson and
               Bharath Hariharan and
               Laurens van der Maaten and
               Li Fei{-}Fei and
               C. Lawrence Zitnick and
               Ross B. Girshick (2016)
\newblock {CLEVR:} {A} Diagnostic Dataset for Compositional Language and Elementary
               Visual Reasoning
\newblock \emph{arXiv} http://arxiv.org/abs/1612.06890.
\end{thebibliography}


\begin{thebibliography}{100} % Beamer does not support BibTeX so references must be inserted manually as below
\bibitem[Hudson, Manning, 2019]{gqa} Drew A. Hudson and
               Christopher D. Manning (2019)
\newblock {GQA:} a new dataset for compositional question answering over real-world images
\newblock \emph{arXiv} http://arxiv.org/abs/1902.09506.
\end{thebibliography}
}
\end{frame}

\begin{frame}
%\frametitle{References}
\footnotesize{
\begin{thebibliography}{99} % Beamer does not support BibTeX so references must be inserted manually as below
\bibitem[Hudson et al, 2018]{mac} Drew A. Hudson and
               Christopher D. Manning (2018)
\newblock Compositional Attention Networks for Machine Reasoning
\newblock \emph{arXiv} https://arxiv.org/abs/1803.03067.
\end{thebibliography}



\begin{thebibliography}{99} % Beamer does not support BibTeX so references must be inserted manually as below
\bibitem[Antol et al, 2015]{vqa} Stanislaw Antol and
               Aishwarya Agrawal and
               Jiasen Lu and
               Margaret Mitchell and
               Dhruv Batra and
               C. Lawrence Zitnick and
               Devi Parikh (2015)
\newblock {VQA:} Visual Question Answering
\newblock \emph{arXiv} http://arxiv.org/abs/1505.00468.
\end{thebibliography}
}


\end{frame}

%------------------------------------------------
\begin{frame}
%\frametitle{References}
\footnotesize{
\begin{thebibliography}{99} % Beamer does not support BibTeX so references must be inserted manually as below
\bibitem[Krishna et al, 2016]{vgenome} Krishna, Ranjay and Zhu, Yuke and Groth, Oliver and Johnson, Justin and Hata, Kenji and Kravitz, Joshua and Chen, Stephanie and Kalantidis, Yannis and Li, Li-Jia and Shamma, David A and others (2016)
\newblock Visual genome: Connecting language and vision using crowdsourced dense image annotations
\newblock \emph{arXiv} https://arxiv.org/abs/1602.07332.
\end{thebibliography}
}


\end{frame}


%------------------------------------------------

\begin{frame}
\Huge{\centerline{Thank you!}}
\end{frame}

%----------------------------------------------------------------------------------------

\end{document} 