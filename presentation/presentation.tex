%%%%%%%%%%%%%%%%%%%%%%%%%%%%%%%%%%%%%%%%%
% Beamer Presentation
% LaTeX Template
% Version 1.0 (10/11/12)
%
% This template has been downloaded from:
% http://www.LaTeXTemplates.com
%
% License:
% CC BY-NC-SA 3.0 (http://creativecommons.org/licenses/by-nc-sa/3.0/)
%
%%%%%%%%%%%%%%%%%%%%%%%%%%%%%%%%%%%%%%%%%

%----------------------------------------------------------------------------------------
%	PACKAGES AND THEMES
%----------------------------------------------------------------------------------------

\documentclass{beamer}

\mode<presentation> {

% The Beamer class comes with a number of default slide themes
% which change the colors and layouts of slides. Below this is a list
% of all the themes, uncomment each in turn to see what they look like.

%\usetheme{default}
%\usetheme{AnnArbor}
%\usetheme{Antibes}
%\usetheme{Bergen}
%\usetheme{Berkeley}
\usetheme{Berlin}
%\usetheme{Boadilla}
%\usetheme{CambridgeUS}
%\usetheme{Copenhagen}
%\usetheme{Darmstadt}
%\usetheme{Dresden}
%\usetheme{Frankfurt}
%\usetheme{Goettingen}
%\usetheme{Hannover}
%\usetheme{Ilmenau}
%\usetheme{JuanLesPins}
%\usetheme{Luebeck}
%\usetheme{Madrid}
%\usetheme{Malmoe}
%\usetheme{Marburg}
%\usetheme{Montpellier}
%\usetheme{PaloAlto}
%\usetheme{Pittsburgh}
%\usetheme{Rochester}
%\usetheme{Singapore}
%\usetheme{Szeged}
%\usetheme{Warsaw}

% As well as themes, the Beamer class has a number of color themes
% for any slide theme. Uncomment each of these in turn to see how it
% changes the colors of your current slide theme.

%\usecolortheme{albatross}
%\usecolortheme{beaver}
%\usecolortheme{beetle}
%\usecolortheme{crane}
%\usecolortheme{dolphin}
%\usecolortheme{dove}
%\usecolortheme{fly}
%\usecolortheme{lily}
%\usecolortheme{orchid}
%\usecolortheme{rose}
%\usecolortheme{seagull}
%\usecolortheme{seahorse}
%\usecolortheme{whale}
%\usecolortheme{wolverine}

%\setbeamertemplate{footline} % To remove the footer line in all slides uncomment this line
%\setbeamertemplate{footline}[page number] % To replace the footer line in all slides with a simple slide count uncomment this line

%\setbeamertemplate{navigation symbols}{} % To remove the navigation symbols from the bottom of all slides uncomment this line
}

\usepackage{graphicx} % Allows including images
\usepackage{booktabs} % Allows the use of \toprule, \midrule and \bottomrule in tables
\usepackage{hyperref}
\usepackage{xcolor}
\usepackage{amsmath}


%----------------------------------------------------------------------------------------
%	TITLE PAGE
%----------------------------------------------------------------------------------------

\title[Research Seminar, Winter 2021-2022]{GQA: A New Dataset for Real-World Visual Reasoning and Compositional Question Answering} % The short title appears at the bottom of every slide, the full title is only on the title page

\author{Shawon Ashraf} % Your name
\institute[IMS, Universität Stuttgart] % Your institution as it will appear on the bottom of every slide, may be shorthand to save space
{
Universität Stuttgart \\ % Your institution for the title page
}
\date{16 December 2021} % Date, can be changed to a custom date

\begin{document}

\begin{frame}
\titlepage % Print the title page as the first slide
\end{frame}

\begin{frame}
\frametitle{Overview} % Table of contents slide, comment this block out to remove it
\tableofcontents % Throughout your presentation, if you choose to use \section{} and \subsection{} commands, these will automatically be printed on this slide as an overview of your presentation
\end{frame}

%----------------------------------------------------------------------------------------
%	PRESENTATION SLIDES
%----------------------------------------------------------------------------------------

%------------------------------------------------
\section{Introduction and Background} % Sections can be created in order to organize your presentation into discrete blocks, all sections and subsections are automatically printed in the table of contents as an overview of the talk
%------------------------------------------------

\begin{frame}
\frametitle{What is GQA?}

\begin{itemize}
    \item Visual Question Answering
        \begin{itemize}
            \item \textbf{Understanding} a Scene (Image) and \textbf{answering questions} on it
        \end{itemize}
    \item Combination of Structural and Semantic Information
        \begin{itemize}
            \item \textbf{Objects} in the image and how they are \textbf{related}
            \item How the objects and relations connect to question text
        \end{itemize}
    \item Scene Graphs
        \begin{itemize}
            \item Data Structure for the required \textbf{structure}
            \item Source: Visual Gnome
        \end{itemize}
\end{itemize}

\end{frame}

\begin{frame}{Why do we need it?}
    \begin{itemize}
        \item \textbf{Lack of diveristy in the labels} in existing datasets (CLEVR, VQA etc.)
        \item Susceptible to \textbf{lucky guesses}
        \item Exitsing approaches (such as CNN, LSTM etc.) towards the problem \textbf{can easily game} these datasets
        \item GQA solves this problem 
    \end{itemize}
\end{frame}

\begin{frame}
\frametitle{Consider this image (Source: GQA \cite{p2})}
\begin{figure}
\includegraphics[width=0.8\linewidth]{gqa_1.png}
\end{figure}
\end{frame}


\begin{frame}
\frametitle{A scene graph of the previous image (Source: GQA \cite{p2})}
\begin{figure}
\includegraphics[width=0.8\linewidth]{scene_graph.png}
\end{figure}
\end{frame}
%------------------------------------------------

\section{Dataset Statistics}

%------------------------------------------------

\section{Results}
%------------------------------------------------

\section{Conclusion}
%------------------------------------------------

%------------------------------------------------
% ----------------------------- Ref Section -----------------------------------------------------------------
\section{References}

\begin{frame}
%\frametitle{References}
\footnotesize{
\begin{thebibliography}{99} % Beamer does not support BibTeX so references must be inserted manually as below
\bibitem[Lian et al, 2021]{p1} Liang, Weixin  and Jiang, Yanhao  and Liu, Zixuan (2021)
\newblock {G}ragh{VQA}: Language-Guided Graph Neural Networks for Graph-based Visual Question Answering
\newblock \emph{Proceedings of the Third Workshop on Multimodal Artificial Intelligence} 79--86.
\end{thebibliography}

\begin{thebibliography}{100} % Beamer does not support BibTeX so references must be inserted manually as below
\bibitem[Hudson, Manning, 2019]{p2} Drew A. Hudson and
               Christopher D. Manning (2019)
\newblock {GQA:} a new dataset for compositional question answering over real-world images
\newblock \emph{arXiv} http://arxiv.org/abs/1902.09506.
\end{thebibliography}
}
\end{frame}

\begin{frame}
%\frametitle{References}
\footnotesize{
\begin{thebibliography}{99} % Beamer does not support BibTeX so references must be inserted manually as below
\bibitem[Zellers et al, 2018]{p3} Zellers, Rowan and Yatskar, Mark and Thomson, Sam and Choi, Yejin (2018)
\newblock Neural Motifs: Scene Graph Parsing with Global Context
\newblock \emph{Conference on Computer Vision and Pattern Recognition}.
\end{thebibliography}


\begin{thebibliography}{99} % Beamer does not support BibTeX so references must be inserted manually as below
\bibitem[Johnson et al, 2016]{p4} Justin Johnson and
               Bharath Hariharan and
               Laurens van der Maaten and
               Li Fei{-}Fei and
               C. Lawrence Zitnick and
               Ross B. Girshick (2016)
\newblock {CLEVR:} {A} Diagnostic Dataset for Compositional Language and Elementary
               Visual Reasoning
\newblock \emph{arXiv} http://arxiv.org/abs/1612.06890.
\end{thebibliography}
}
\end{frame}

\begin{frame}
%\frametitle{References}
\footnotesize{
\begin{thebibliography}{99} % Beamer does not support BibTeX so references must be inserted manually as below
\bibitem[Santoro et al, 2017]{p5} Adam Santoro and
               David Raposo and
               David G. T. Barrett and
               Mateusz Malinowski and
               Razvan Pascanu and
               Peter W. Battaglia and
               Timothy P. Lillicrap (2017)
\newblock A simple neural network module for relational reasoning
\newblock \emph{arXiv} http://arxiv.org/abs/1706.01427.
\end{thebibliography}

}
\end{frame}

%------------------------------------------------


%------------------------------------------------

\begin{frame}
\Huge{\centerline{The End}}
\end{frame}

%----------------------------------------------------------------------------------------

\end{document} 