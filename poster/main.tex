\documentclass[25pt, a0paper, portrait, margin=0mm, innermargin=15mm,blockverticalspace=15mm, colspace=15mm, subcolspace=8mm]{tikzposter}

\input{packages}
\input{apparence}

\title{\parbox{1700pt}{PUT TITLE HERE - NOTE THAT IF IT'S VERY LONG, IT SPLITS AUTOMATICALLY}}

\author{Prénom Nom, Prénom Nom}
\institute{Univ. Grenoble Alpes, CNRS, Grenoble INP, LIG, 38000 Grenoble, France}

\usetitlestyle[]{sampletitle}
\setlength{\columnseprule}{0.4pt}
\addbibresource{Biblio/Biblio.bib}

%%%%%%%%%%%%%%%%%%%%%%%%%%%%%%%%%%%%%%%%%%%%%%%%%%%%%%%%%%%
\begin{document}
\maketitle

% --- Lignes entre les colonnes
\draw[DarkerGrey, line width=2mm, loosely dotted] (-13,35) -- (-13,-45); 
\draw[DarkerGrey, line width=2mm, loosely dotted] (15,35) -- (15,-30);

%------------------------------------------------------------------------------
% --------------------- CORPS DU POSTER ---------------------
\begin{columns}
\column{1}
\begin{subcolumns}
    %%%%%%%%%%%% COLONNE 1 %%%%%%%%%%%%%
    \subcolumn{.33}
   
    % ---------------------------------MOTIVATIONS ----------
        \block{\textsc{Motivations}}{\lipsum[1]}
    

    % --------------------------------- PARTIE1----------
        \block{\textsc{Partie 1}}{\begin{itemize}
            \item item 1
            \item item 2
            \item item 3
        \end{itemize}
        
        \vspace{0.5cm}
        
        \underline{liste 2}: \begin{itemize}[label=-]
            \item item;
            \item item;
            \item item;
            \item item;
        \end{itemize} \vspace{-2cm}}
        
      % --------------------------------- PARTIE 2  ----------    
    \block{\textsc{Partie 2}}{
        
        \tikzstyle{na}=[baseline=-.25ex]
        \tikzstyle{every picture}+=[remember picture]
        \vspace{0.5cm}
        
        \innerblock{Avec titre}{\begin{tikzpicture}
\node[draw=Orange,fill=Orange,text=G-lig,text width=6cm, rounded corners, text centered] (item1) at (2,4) {item1};

\node[draw=Orange,fill=Orange,text=G-lig,text width=6cm,rounded corners, text centered] (item2) at (2,0) {item2};

\node[draw=Gray,text width=5cm,fill=Gray,rounded corners, text centered, text=FontColor, text=G-lig] (item3) at (10,4) {item3};

\node[draw=LightRed,text width=5cm,fill=LightRed,text=G-lig,rounded corners, text centered] (item4) at (10,-3) {item4};

\node[draw=LightBlue,text width=5cm,text=G-lig,fill=LightBlue,rounded corners, text centered] (item5) at (10,0) {item5};

\node[draw=LightGreen,fill=LightGreen,text=G-lig,text width=5cm,rounded corners, text centered] (item6) at (18,0) {item6};

\draw[->, line width=1mm, draw=FontColor] (item1) to (item2);
\draw[->,line width=1mm, draw=FontColor] (item2) to (item3);
\draw[->,line width=1mm,draw=FontColor] (item2) to (item4);
\draw[->,line width=1mm,draw=FontColor] (item2) to (item5);
\draw[->,line width=1mm,draw=FontColor] (item3) to (item6);
\draw[->,line width=1mm,draw=FontColor] (item5) to (item6);
\draw[->,line width=1mm,draw=FontColor] (item4) to (item6);

\end{tikzpicture}


        } 
        \vspace{0.5cm}
        \lipsum[1]
            
          \vspace{-2cm}}
          
    %%%%%%%%%%%%%%% COLONNE 2 %%%%%%%%%%%%%
    \subcolumn{.33}
    
        % ---------------------------------  PARTIE 3----------
        \block{\textsc{Partie 3}}{\innerblock{}{
        \input{figures/figure2}} \lipsum[2]
        \vspace{-2cm}}
            
        \note[targetoffsetx = 3cm, targetoffsety = 1cm, angle = 20, connection]{\textbf{exemple de note}}

        
    % --------------------------------- PARTIE 4  ----------
        \block{\textsc{Partie 4}}{ \lipsum[3]
        \vspace{-2cm}}
    
  
 % ------------------------------------ PARTIE 5------------
        \block{\textsc{Partie 5}}{\lipsum[1]  
        
        \input{table/table} 
            \vspace{-2cm}}

 
 %%%%%%%%%%%%%%%%%%%%%%%% COLONNE 3 %%%%%%%%%%%%%%%%%%%%%%%%%   
    \subcolumn{.33}

    % ---------------------------------------------
        \block{\textsc{Conclusion}}{
        \lipsum[1]
        
        \vspace{0.5cm}
        \lipsum[2] \cite{ref1}
        \vspace{0.5cm}}

    % ----------------------------------------------
        \block{}{
        \vspace{3cm}
            \textbf{SPA}: Société Protectrice des Animaux $\bullet$
            \textbf{CAF}: Caisse d'Allocations Familiales $\bullet$
            \textbf{WER}: Word Error Rate }   
    
    \end{subcolumns}
    

    % ---------------------------------  BIBLIO ----------
    \block{}{\vspace{1cm}
       \printbibliography}
       \end{columns}

% ----------------- Ligne colorée à la fin du document -------------
\node [above right, text=white,outer sep=45pt,minimum width=\paperwidth, align=center, draw, fill=titledarkcolor, color=point-lig] at (-43.6,-61) { \textcolor{white}{\normalsize Contact: prenom.nom@univ-grenoble-alpes.fr}};

\end{document}
